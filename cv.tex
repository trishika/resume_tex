\documentclass[11pt,a4paper]{moderncv}

\usepackage{verbatim}
\usepackage{xcolor}

% moderncv themes
\moderncvtheme[blue]{classic}
% optional argument are 'blue' (default), 'orange', 'red', 'green', 'grey' and 'roman' (for roman fonts, instead of sans serif fonts)

% character encoding
\usepackage[utf8]{inputenc}

% adjust the page margins
\usepackage[scale=0.94]{geometry}

% Itemize item
\usepackage{enumitem}
%\setlist{leftmargin=5.5mm}
\newlist{myitemize}{itemize}{3}
\setlist[myitemize,1]{noitemsep,nolistsep,label=\textbullet,leftmargin=2em}
\setlist[myitemize,2]{noitemsep,nolistsep,label=$\rightarrow$,leftmargin=1em}
\setlist[myitemize,3]{noitemsep,nolistsep,label=$\diamond$}

% ajust the width of the column with the dates
\setlength{\hintscolumnwidth}{3cm}

% CV TYPE
\usepackage{ifthen}

\ifdef{\Lang}
{}{\def\Lang{fr}}
%\def\Lang{en}
\ifthenelse{\equal{\Lang}{en}}
	{\newcommand{\T}[2]{#1}}
	{\newcommand{\T}[2]{#2}}
%\newcommand{\T}[2][]{\ifthenelse{\equal{\LangCond}{en}}{#1}{#2}}

\ifdef{\Type}
{}{\def\Type{Short}}
%\def\Type{Long}
\ifthenelse{\equal{\Type}{Long}}
	{\newcommand{\LongText}[1]{#1}}
	{\newcommand{\LongText}[1]{}}
\ifthenelse{\equal{\Type}{Short}}
	{\newcommand{\ShortText}[1]{#1}}
	{\newcommand{\ShortText}[1]{}}
%\newcommand{\LongText}[1]{\ifthenelse{\equal{\TypeCond}{Long}}{#1}{}}

% HEADER
\firstname{Aurélien}
\familyname{Chabot}
\title{\T{Linux Software Engineer}{Ingénieur de développement informatique}}
%\address{40 rue alphonse penaud}{75020 Paris}
%\mobile{(+33)6 45 90 58 47}
\email{aurelien@chabot.fr}
\homepage{www.aurelienchabot.fr}
\extrainfo{\T{French - 26 years old}{Né le 28 août 1987 - 26 ans}}
\photo{photo.jpg}

\begin{document}
\maketitle

\vspace*{-12mm}

\section{\textsc{\T{Studies}{Formation}}}

	\cventry{2008 - 2011}
	{\T{Engineering diploma ( Master's degree in Engineering )}{Diplômé ingénieur en informatique}}
	{INP ENSEEIHT \T{( top ranking engineering school )}{( École d'ingénieur )}}
	{Toulouse\T{ ( France )}{}}{}
	{
		\T{Specialized in Computer Science and Applied Mathematics}{\underline{Filière} : Informatique et mathématiques appliquées}.
		\LongText{
			\newline{}
			\T{\underline{Subjects} : Theory of computation, Software engineering, Computer architecture, Concurrent computing, Network architecture, Compilation.}
			{\underline{Enseignements Suivis} : Théorie informatique, Génie logiciel, Architecture des ordinateurs, Systèmes concurrents, Intergiciel, Système temps réel, Architecture des réseaux, Compilation, Analyse de données, Base de données, Technologie web.}
		}
	}

	\cventry{Sept - \T{Dec}{Déc} 2010}
	{\T{Master degree in computer science}{Master en informatique}}
	{HKU ( Hong Kong University )}{}{}
	{
		\T{Exchange student in undergraduate and postgraduate courses}{Étudiant en échange}.
		\LongText{ \newline{}
			\T{
				\underline{Subjects} : Multimedia computing and application, Computer and network security,Embedded Systems and
				pervasive computing, Implementation, Testing and Maintenance of Software Systems, Wireless, Image processing
				and computer vision.
			}{
				\underline{Enseignements Suivis} : Multimedia et application, Sécurité des réseaux,
				Système embarqué et informatique ubiquitaire, Implémentation, Test et Maintenance des systèmes informatiques,
				Réseaux sans fil, Traitement d'Images et Vision par Ordinateur.
			}
		}
	}

	\cventry{2005 - 2008}
	{ \T{Three year intensive undergraduate course in preparatory classes
	for competitive entrance exams into national engineering schools}
	{Classes préparatoires scientifiques ( MPSI/MP )}}
	{Lycée Faidherbe, Lille \T{( France )}{( 59 )}}{}{}
	{\T{\underline{Subjects} : Mathematics, Physics and Computing}
	{\underline{Enseignements Suivis} : Mathématique, Physique and Informatique}}

	\cventry{2005}
	{\T{Scientific Baccalaureate with distinction \small ( French equivalent of 'A' levels in Mathematics, Physics and Biology )}
	{Baccalauréat Scientifique avec mention bien}}
	{Lycée Robespierre, Arras \T{( France )}{( 62 )}}
	{}{}{}{}


\section{\textsc{\T{Experiences}{Expériences}}}

	\subsection{\textsc{\T{Job}{Emplois}}}

		\cventry{\T{Nov 2014 - now}{Depuis Nov 2014}}
		{\T{Software engineer, working on android embedded system for cars}
		{Ingénieur de développement pour des autoradios sur android}}
		{Parrot}{Paris}{}{}

		\LongText{
			\cvline{}{
				\T{Development of the android platform use on multiple OEM product}
				{Développement de la platform android utilisé par plusieur produit OEM}
			}
		}

		\LongText{
			\cvline{}{
				\small
				\T{Technical environment}{Environnement technique} :
				\begin{myitemize}
					\item Java, C/C++ ;
					\item Android ;
					\item \T{Bluetooth protocol}{Protocol bluetooth}.
				\end{myitemize}
			}
		}

		\cventry{\T{Oct 2011 - Nov 2014}{Oct 2011 - Nov 2014}}
		{\T{Consulting software engineer, specialized in image processing and Linux embedded systems}
		{Ingénieur de développement, spécialisé dans le traitement d'image et les systèmes embarqués Linux}}
		{OpenWide}{Paris}{}{}

		% Jun 2014 - Nov 2014
		\cventry{\T{6 month}{6 mois}}
		{\T{Contractor, OTT video streaming packaging}
		{Ingénieur de développement consultant sur un logiciel de trans-packaging de flux video}}
		{Anevia}{Paris}{}{}

		\LongText{
			\cvline{}{
				\small
				\T{Technical environment}{Environnement technique} :
				\begin{myitemize}
					\item PHP, Python, C/C++ ;
					\item Postgresql ;
					\item \T{Debian packaging}{Generation de paquet Debian}.
				\end{myitemize}
			}
		}

		% Nov 2013 - May 2014
		\cventry{\T{10 month}{10 mois}}
		{\T{Development on a CCTV analysis software for subway interchange analysis}
		{Developpement sur un logicel d'analyse des échanges de voyageur entre le quai et le métro par vidéos surveillance.}}
		{OpenWide}{Paris}{}{}

		\LongText{
			\T{Software builds to optimize subway parking time in station in the case of a driveless subway.
			The system is linked to the control system of the subway and allow to impact in real time the subway parking time.}
			{Logiciel connecté en temps réel au système de contrôle du métro automatique afin d'optimiser
			l'arrêt des trains en station selon les échanges voyageurs.}

			\cvline{}{
				\small
				\T{Technical environment}{Environnement technique} :
				\begin{myitemize}
					\item C++, Python, QT ;
					\item Thrift, Boost, RobotFramework.
				\end{myitemize}
			}
		}

		% Sept 2012 - Oct 2013
		\cventry{\T{1 year}{1 ans}}
		{\T{Contractor, working on multimedia embedded system for cars}
		{Ingénieur de développement consultant pour les équipements multimédias pour l'automobile}}
		{Parrot}{Paris}{}{}

		\LongText{
			\cvline{}{
				\T{Integration and development of a Linux OEM product dedicated to bluetooth, wifi and usb connectivity.}
				{Intégration et développement d'un produit OEM Linux dédié à la connectivité bluetooth, wifi et usb.}
			}
		}

		\LongText{
			\cvline{}{
				\small
				\T{Technical environment}{Environnement technique} :
				\begin{myitemize}
					\item C/C++ ;
					\item \T{Bluetooth protocol}{Protocol bluetooth}.
				\end{myitemize}
			}
		}

		% Oct 2011 - Aug 2012
		\cventry{\T{1 year}{1 ans}}
		{\T{Development on a CCTV analysis software to detect camera defect}
		{Developpement sur un logiciel d'analyse de vidéos surveillance pour la détection de défauts des cameras}}
		{OpenWide}{Paris}{}{}

		\LongText{
			\cvline{}{
				\T{Software builds to maintain the reliability of big CCTV installation by doing daily check on CCTV
				records. The camera state is computed from various criteria : blur image, over/under-exposition, moved camera...}
				{Logiciel chargé d'assurer la fiabilité de grands parcs de vidéo surveillance en effectuant des contrôles
				journaliers sur les enregistrements vidéos. L'état de la caméra est déterminé grâce à différents critères : image floue,
				sur-exposition ou encore décadrage de la caméra.}
			}
		}

		\LongText{
			\cvline{}{
				\small
				\T{Technical environment}{Environnement technique} :
				\begin{myitemize}
					\item \T{C++ QT}{C++ QT} ;
					\item \T{OpenCV library for camera image processing}{Bibliothèque OpenCV pour l'analyse des images récupérées sur les caméras} ;
					\item \T{Several SDK for camera recorder}{Plusieurs SDK d'enregistreurs vidéos}.
				\end{myitemize}
			}
		}

	\subsection{\textsc{\T{Internchip}{Stages}}}

		\cventry{\T{March-Sept}{Mars-Sept} 2011}
		{\T{Internship in image processing development}{Stage en développement pour le traitement d'image}}
		{OpenWide}{Paris}{}
		{\T{C++ development for CCTV image processing. Development with QT and OpenCV.}{Développement en C++ de traitement des images pour la vidéo-surveillance.}}

		\cventry{\T{Summer}{Été} 2010}
		{\T{Internship in an image processing laboratory}{Stage dans un laboratoire de traitement d'image}}
		{\T{South Dakota State University ( US )}{Université d'État du Dakota du Sud ( États-Unis )}}{}{}
		{\T{Development of a tools to apply an interpolation algorithms on a satellite database with matlab.}
		{Développement d'un outil d'interpolation de données satellites avec matlab.}}

		\cventry{\T{Summer}{Été} 2009}
		{\T{Internship in web development}{Stage en développement web}}{JNOV}{}{}
		{\T{Development in html, php, sql and javascript. Use of library and api : Google Map, fpdf, Jquery...}
		{Développement en html, php, sql et javascript. Utilisation de bibliothèque et api : Google Map, fpdf, Jquery...}}


	\subsection{\textsc{\T{Personnal projects}{Projets personnels}}}

		\cventry{DroidUPnP}
		{\T{Android application}{Application Android}}
		{}{}{}
		{\T{Allow the control of UPnP compatible devices and the sharing of local multimedia content with UPnP.}
		{Permet le contrôle d'appareil compatible UPnP et le partage du contenu multimédia local.}}

	\LongText{
		\cventry{Home-Pi}
		{\T{A Home domotique domotique service on top of a raspberrypi.}
		{Service de domotique pour la maison déployable sur raspberrypi.}}
		{}{}{}
		{\T{By controlling a temperature sensor (through I2C) and a radio emmiter (through GPIO),
		home heaters and lights are monitored, controled and scheduled from a phone or a computer.}
		{Grace au contrôle d'un capteur de température (via I2C) et d'un émetteur radio (via GPIO),
		les chauffages et lumières peuvent être contrôlés et programmés depuis un téléphone ou un ordinateur.}}
	}

	%\LongText{
	%	\subsection{\textsc{\T{Study Project}{Projets d'Études}}}

			%\cventry{2009 - 2010}
			%{\T{C++ Project ( in pair )}{Projet en C++ ( en binôme )}}
			%{}{}{}
			%{\T{Realization of a cryptanalyseur to decrypt message, crypt with simple permutation key, by using statistic analysis.}
			%{Réalisation d'un cryptanalyseur pour chiffrer des messages, cryptés avec une clé simple a permutation,
			%en utilisant l'analyse statistique.}}

			%\cventry{2008 - 2009}
			%{\T{C Project ( single )}{Projet en C ( individuel )}}
			%{}{}{}
			%{
			%	\T{Realisation of a calculator for arithmetic expressions in Polish suffixed notation.}
			%	{Réalisation d'un calculateur d'expressions arithmétiques en notation polonaise suffixée.}
			%	\begin{itemize}
			%		\item \T{Realisation of a linear structure ( stack ) module handling abstract types}{Réalisation d'un module de gestion de structures linéaires ( piles ) manipulant des types abstraits} ;
			%		\item \T{Memory management by allocation and release}{Gestion de la mémoire par allocation et libération} ;
			%		\item \T{Unit testing on every function}{Test unitaire sur chaque fonction}.
			%	\end{itemize}
			%}

			%\cventry{2008 - 2009}
			%{\T{Java Project ( in pairs )}{Projet en Java ( en binôme )}}
			%{}{}{}
			%{
			%	\T{Realisation of a software, with a graphic user interface, permitting to generate pictures in a scene
			%	containing objects with a ray tracer algorithm.}
			%	{Réalisation d'un programme, avec une interface graphique, permettant de générer des images
			%	grâce à un algorithme de lancer de rayons dans une scène contenant des objets géométriques.}
			%	\begin{myitemize}
			%		\item \T{Elaboration of UML diagrams}{Élaboration de diagrammes de modélisation UML} ;
			%		\item \T{Development in respecting MVC pattern}{Développement de l'application en respectant le modèle MVC} ;
			%		\item \T{Management of pictures recording in PPM format thus scenes and points of view recording in XML format ;
			%				XML files syntax respecting a provided DTD file}
			%				{Gestion de l'enregistrement des images au format PPM ainsi que de la scène et des points de vues au format XML ;
			%				la syntaxe des fichiers XML respectant une DTD fournie} ;
			%		\item \T{Tests of the different functions with JUnit}{Élaboration de tests unitaires à l'aide de JUnit}.
			%	\end{myitemize}
			%}
	%}

\section{\textsc{Langues}}

	\T{\cvlanguage{French}{Native speaker}{}}{}

	\cvlanguage{\T{English}{Anglais}}{\T{Working Proficient}{Niveau professionnel}, \textbf{TOEIC 934}}{}

	%\cvlanguage{\T{Japanese \& Spanish}{Japonais \& Espagnol}}{\T{Elementary level}{Notion}}{}

\section{\textsc{\T{IT Skills}{Compétences}}}

	\cvline{\T{Programmings}{Programmation}}{C++, C, Java, Python, Bash, Fortran, CamL, Matlab}
	\cvline{Web}{HTML, CSS, PHP, SQL, JavaScript, JEE}
	\cvline{Libraries / \T{Tools}{Outils}}{Android, QT, OpenCV, Buildroot, Boost}
	\cvline{\T{Version Control}{Gestion de version}}{Git, SVN}


\LongText{
\section{\textsc{\T{Interest}{Loisirs}}}

	\cvline{Sports}{\T{Regular training of swimming and course}{Natation, Course pédestre}.}
	\cvline{\T{Reading}{Lecture}}{\T{Fiction novel, international news}{Roman fiction, Actualité internationale}.}
}

\end{document}
