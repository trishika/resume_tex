\documentclass[11pt,a4paper]{moderncv}

\usepackage{verbatim}
\usepackage{xcolor}

% moderncv themes
\moderncvtheme[blue]{classic}
% optional argument are 'blue' (default), 'orange', 'red', 'green', 'grey' and 'roman' (for roman fonts, instead of sans serif fonts)

% character encoding
\usepackage[utf8]{inputenc}

% Itemize item
\usepackage{enumitem}
%\setlist{leftmargin=5.5mm}
\newlist{myitemize}{itemize}{3}
\setlist[myitemize,1]{noitemsep,nolistsep,label=\textbullet,leftmargin=2em}
\setlist[myitemize,2]{noitemsep,nolistsep,label=$\rightarrow$,leftmargin=1em}
\setlist[myitemize,3]{noitemsep,nolistsep,label=$\diamond$}

% ajust the width of the column with the dates
\setlength{\hintscolumnwidth}{3cm}

% CV TYPE
\usepackage{ifthen}

\ifdef{\Lang}
{}{\def\Lang{fr}}
%\def\Lang{en}
\ifthenelse{\equal{\Lang}{en}}
	{\newcommand{\T}[2]{#1}}
	{\newcommand{\T}[2]{#2}}
%\newcommand{\T}[2][]{\ifthenelse{\equal{\LangCond}{en}}{#1}{#2}}

\ifdef{\Type}
{}{\def\Type{Short}}
%\def\Type{Long}
\ifthenelse{\equal{\Type}{Long}}
	{\newcommand{\LongText}[1]{#1}}
	{\newcommand{\LongText}[1]{}}
\ifthenelse{\equal{\Type}{Short}}
	{\newcommand{\ShortText}[1]{#1}}
	{\newcommand{\ShortText}[1]{}}
%\newcommand{\LongText}[1]{\ifthenelse{\equal{\TypeCond}{Long}}{#1}{}}


% adjust the page margins
\LongText{
	\usepackage[scale=0.93]{geometry}
}
\ShortText{
	\usepackage[scale=0.96]{geometry}
}


% HEADER
\firstname{Aurélien}
\familyname{Chabot}
\title{\T{Linux Software Engineer}{Ingénieur de développement informatique}}
%\address{40 rue alphonse penaud}{75020 Paris}
%\mobile{(+33)6 45 90 58 47}
\email{aurelien@chabot.fr}
\homepage{www.aurelienchabot.fr}
\extrainfo{\T{French - 28 years old}{Né le 28 août 1987 - 28 ans}}
\LongText{
	\photo[60pt][0.4pt]{photo.jpg}
}
\ShortText{
	\photo[50pt][0.4pt]{photo.jpg}
}

\begin{document}
\maketitle

\vspace*{-12mm}

\LongText{
	\medskip
}

\LongText{
	\setlength{\parskip}{0.15em}
}
\ShortText{
	\setlength{\parskip}{0.1em}
}

\section{\textsc{\T{Studies}{Formation}}}

	\cventry{2008 - 2011}
	{\T{Engineering diploma ( Master's degree in Engineering )}
	{Diplômé ingénieur en informatique}}
	{INP ENSEEIHT \T{(top ranking engineering school)}{(École d'ingénieur)}}
	{Toulouse\T{ ( France )}{}}{}
	{
		\T{Specialized in Computer Science and Applied Mathematics}
		{\underline{Filière} : Informatique et mathématiques appliquées}.
		\LongText{
			\newline{}
			\T{\underline{Subjects} : Theory of computation, Software engineering,
			Computer architecture, Concurrent computing, Network architecture, Compilation.}
			{\underline{Enseignements Suivis} : Théorie informatique, Génie logiciel,
			Architecture des ordinateurs, Systèmes concurrents, Intergiciel,
			Système temps réel, Architecture des réseaux, Compilation, Analyse de données,
			Base de données, Technologie web.}
		}
	}

	\cventry{Sept - \T{Dec}{Déc} 2010}
	{\T{Master degree in computer science}{Master en informatique}}
	{HKU (Hong Kong University)}{}{}
	{
		\T{Exchange student in undergraduate and postgraduate courses}{Étudiant en échange}.
		\LongText{ \newline{}
			\T{
				\underline{Subjects} : Multimedia computing and application,
				Computer and network security,Embedded Systems and pervasive computing,
				Implementation, Testing and Maintenance of Software Systems, Wireless,
				Image processing and computer vision.
			}{
				\underline{Enseignements Suivis} : Multimedia et application,
				Sécurité des réseaux, Système embarqué et informatique ubiquitaire,
				Implémentation, Test et Maintenance des systèmes informatiques,
				Réseaux sans fil, Traitement d'Images et Vision par Ordinateur.
			}
		}
	}

	\cventry{2005 - 2008}
	{ \T{Three year intensive undergraduate course in preparatory classes
	for competitive entrance exams into national engineering schools}
	{Classes préparatoires scientifiques (MPSI/MP)}}
	{Lycée Faidherbe, Lille \T{(France)}{(59)}}{}{}
	{\T{\underline{Subjects} : Mathematics, Physics and Computing}
	{\underline{Enseignements Suivis} : Mathématique, Physique and Informatique}}

	\cventry{2005}
	{\T{Scientific Baccalaureate with distinction \small (French equivalent of 'A' levels
	in Mathematics, Physics and Biology)}
	{Baccalauréat Scientifique avec mention bien}}
	{Lycée Robespierre, Arras \T{(France)}{(62)}}
	{}{}{}{}

\LongText{
	\medskip
}

\section{\textsc{\T{Experience}{Expérience}}}

	\subsection{\textsc{\T{Jobs}{Emplois}}}

		\cventry{\T{Sept 2016 - now}{Depuis Sept 2016}}
		{\T{Software engineer}
		{Ingénieur de développement}}
		{Corvil}{Dublin}{}{
				\T{Design, development and evolution of the broad range of Corvil Analytics 
				Plugins for all major worldwide financial, middleware and enterprise data protocols.}
				{Développement et évolution des plugin d'analyse de Corvil pour les principaux 
				protocols des systèmes financiers et des applications d'entreprises.}
		}

		\LongText{
			\cvline{}{
				\small
				\addtolength{\leftskip}{0.5em}
				\underline{\T{Technical environment}{Environnement technique}} :
				C++, boost, Linux
			}
		}

		\cventry{\T{Dec 2014 - Sept 2016}{Dec 2014 - Sept 2016}}
		{\T{Software engineer, Android infotainment system for cars}
		{Ingénieur de développement, Systèmes multimédia pour l'automobile sur Android}}
		{Parrot Automotive}{Paris}{}{
				\T{Android application development, developing middleware
				interfaces to component such as radio and Bluetooth as well as
				graphical views and behaviors of the media, telephony and system applications.}
				{Développement d'applications Android constituant le système multimédia ainsi
				que de middleware pour s'interfacer avec des fonctionnalités du système
				tel que la radio et le Bluetooth}
		}

		\LongText{
			\cvline{}{
				\small
				\addtolength{\leftskip}{0.5em}
				\underline{\T{Technical environment}{Environnement technique}} :
				Android, Java, C++, Android Studio, Vim, Git, Linux
			}
		}

		% Jun 2014 - Nov 2014
		\cventry{\T{Jun 2014 - Nov 2014}{Juin 2014 - Nov 2014}}
		{\T{Software engineer, OTT video streaming packaging (contractor for OpenWide)}
		{Ingénieur de développement, packaging de flux video pour les protocols OTT
		(prestation pour OpenWide)}}
		{Anevia}{Paris}{}{
			\T{Development for a server providing video content with OTT protocols (DASH, HLS...).}
			{Développement pour serveur fournissant du contenu vidéo avec les protocoles OTT (DASH, HLS...)}}

		\LongText{
			\cvline{}{
				\small
				\T{Migration to PostgreSQL (from SQLite) by writing
				database management tools in python up to the web user interface
				in PHP. Worked also on database request optimization
				as well as load balancing and fail over strategy development.}
				{Migration vers PostgreSQL (depuis SQLite) en développent des outils de
				gestion de base de données en python. Optimisation des requêtes SQL. Développement d'une nouvelle architecture de load balancing et de fail over.}
			}
		}

		\LongText{
			\cvline{}{
				\small
				\addtolength{\leftskip}{0.5em}
				\underline{\T{Technical environment}{Environnement technique}} :
				Python, C++, PostgreSQL, PHP, Javascript, Debian, Vim, Git/SVN, Linux
			}
		}

		% Nov 2013 - May 2014
		\cventry{\T{Nov 2013 - May 2014}{Nov 2013 - Mai 2014}}
		{\T{Software engineer, Image processing and Linux system development}
		{Ingénieur de développement, Traitement d'image et dévelopement linux}}
		{OpenWide}{Paris}{}{
			\T{Project to optimize the park time of automatic subway with video
			analysis and real time communication with the command center.}
			{Projet d'optimisation des temps d'arrêt en station d'un métro
			automatique par analyse de vidéo et communication en temps réel avec
			le centre de commande.}
		}

		\LongText{
			\cvline{}{
				\small
				\T{Image processing optimization and adaptation of existing algorithms.
				Architecture and development of the communication stack between the on site
				server and the centralized one, as well as with the command center.
				Architecture and development of a global control application of the system in
				Qt. Project and team leading.}
				{Optimisation et adaptation d'algorithmes existant. Architecture et 
				développement de la communication entre les serveurs en station et le 
				serveur centrale ainsi qu'avec le serveur de commande du métro.
				Architecture et développement d'une application de controle global 
				du système en Qt. Management du projet et gestion d'équipe.}
			}
		}

		\LongText{
			\cvline{}{
				\small
				\addtolength{\leftskip}{0.5em}
				\underline{\T{Technical environment}{Environnement technique}} :
				C++, OpenCV, QT, Python, Thrift, Boost, RobotFramework, Vim, Git, Linux
			}
		}

		\LongText{
			\newpage
		}

		% Sept 2012 - Oct 2013
		\cventry{\T{Sept 2012 - Oct 13}{Sept 2012 - Oct 13}}
		{\T{Software engineer, Multimedia embedded system for cars (contractor for OpenWide)}
		{Ingénieur de développement, System embarquée multimédia pour l'automobile}}
		{Parrot}{Paris}{}{
			\T{Integration and development of a module dedicated to Bluetooth, Wifi and USB
			connectivity acting as a provider of service to the main infotainment system.}
			{Intégration et développement d'un produit Linux fournissant
			la connectivité bluetooth, wifi et usb au système multimédia de la voiture.}
		}

		\LongText{
			\cvline{}{
				\small
				\T{Embedded Linux integration. Development of specific client feature into the
				core application in C++. Handling of client technical request and feedback.}
				{Integration pour système linux embarquée. Développement de fonctionnalité
				spécifique aux clients et support technique pour les requêtes et retours client.}
			}
		}

		\LongText{
			\cvline{}{
				\small
				\addtolength{\leftskip}{0.5em}
				\underline{\T{Technical environment}{Environnement technique}} :
				C/C++, Bash, Python, Vim, Git, Linux
			}
		}

		% Oct 2011 - Aug 2012
		\cventry{\T{Sept 2011 - Aug 2012 }{Sept 2011 - Août 2012 }}
		{\T{Software engineer, Image processing and Linux system development}
		{Ingénieur de développement, Traitement d'image et développement linux}}
		{OpenWide}{Paris}{}{
			\T{Software helping maintaining the reliability of big CCTV
			network by doing daily check on recording.}
			{Logiciel chargé d'assurer la fiabilité de grands parcs de vidéo surveillance en effectuant des contrôles journaliers sur les enregistrements vidéos.}
		}

		\LongText{
			\cvline{}{
				\small
				\T{Image processing optimization and adaptation of existing algorithms.
				Development of the JSON/REST communication stack between the treatment
				server and web server.}
				{Amélioration de la fiabilité des algorithmes d'analyse d'image dans les
				différents environnement rencontré. Développement d'une stack de communication
				JSON/REST pour échanger les données entre le serveur applicatif et le serveur
				web.}
			}
		}

		\LongText{
			\cvline{}{
				\small
				\addtolength{\leftskip}{0.5em}
				\underline{\T{Technical environment}{Environnement technique}} :
				C++, OpenCV, Vim, Git, Linux
			}
		}

	\subsection{\textsc{\T{Internships}{Stages}}}

		\cventry{\T{Mar - Aug}{Mars - Août} 2011}
		{\T{Internship in image processing development}{Stage en développement pour le traitement d'image}}
		{OpenWide}{Paris}{}
		{\T{C++ development for CCTV image processing. Development with QT and OpenCV.}
		{Développement en C++ de traitement des images pour la vidéo-surveillance.}}

		\cventry{\T{Summer}{Été} 2010}
		{\T{Internship in an image processing laboratory}
		{Stage en laboratoire de traitement d'image}}
		{\T{South Dakota State University (US)}
		{Université d'État du Dakota du Sud (États-Unis)}}{}{}
		{\T{Development of a tools to apply an interpolation
		algorithms on a satellite database with matlab.}
		{Développement d'un outil d'interpolation de données satellites avec matlab.}}

		\LongText{
			\cventry{\T{Summer}{Été} 2009}
			{\T{Internship in web development}{Stage en développement web}}{JNOV}{}{}
			{\T{Development in html, php, sql and javascript.
			Use of library and api : Google Map, fpdf, Jquery...}
			{Développement en html, php, sql et javascript.
			Utilisation de bibliothèque et api : Google Map, fpdf, Jquery...}}
		}

	\subsection{\textsc{\T{Personnal Projects}{Projets personnels}}}

		\cvline{DroidUPnP}
		{\T{\textbf{Android application} to control UPnP
		device and share local multimedia content.}
		{\textbf{Application Android} pour controller les appareils
		UPnP et partager les contenus multimedia locaux.}}

	\LongText{
		\cvline{Home-Pi}
		{\T{\textbf{A Home domotique service on top of raspberrypi.}
		By controlling multiple temperature sensors (DHT11 + ESP8266) and a radio emmiter (through GPIO),
		home heaters and lights are monitored, controled and scheduled from a phone
		or a computer.}
		{\textbf{Service domotique pour raspberrypi.}
		Grace au contrôle d'un capteur de température (via I2C) et d'un émetteur radio
		(via GPIO), les chauffages et lumières peuvent être contrôlés
		et programmés depuis un téléphone ou un ordinateur.}}
	}

	%\LongText{
	%	\subsection{\textsc{\T{Study Projects}{Projets d'Études}}}

			%\cventry{2009 - 2010}
			%{\T{C++ Project ( in pair )}{Projet en C++ ( en binôme )}}
			%{}{}{}
			%{\T{Realization of a cryptanalyseur to decrypt message, crypt with simple permutation key, by using statistic analysis.}
			%{Réalisation d'un cryptanalyseur pour chiffrer des messages, cryptés avec une clé simple a permutation,
			%en utilisant l'analyse statistique.}}

			%\cventry{2008 - 2009}
			%{\T{C Project ( single )}{Projet en C ( individuel )}}
			%{}{}{}
			%{
			%	\T{Realisation of a calculator for arithmetic expressions in Polish suffixed notation.}
			%	{Réalisation d'un calculateur d'expressions arithmétiques en notation polonaise suffixée.}
			%	\begin{itemize}
			%		\item \T{Realisation of a linear structure ( stack ) module handling abstract types}{Réalisation d'un module de gestion de structures linéaires ( piles ) manipulant des types abstraits} ;
			%		\item \T{Memory management by allocation and release}{Gestion de la mémoire par allocation et libération} ;
			%		\item \T{Unit testing on every function}{Test unitaire sur chaque fonction}.
			%	\end{itemize}
			%}

			%\cventry{2008 - 2009}
			%{\T{Java Project ( in pairs )}{Projet en Java ( en binôme )}}
			%{}{}{}
			%{
			%	\T{Realisation of a software, with a graphic user interface, permitting to generate pictures in a scene
			%	containing objects with a ray tracer algorithm.}
			%	{Réalisation d'un programme, avec une interface graphique, permettant de générer des images
			%	grâce à un algorithme de lancer de rayons dans une scène contenant des objets géométriques.}
			%	\begin{myitemize}
			%		\item \T{Elaboration of UML diagrams}{Élaboration de diagrammes de modélisation UML} ;
			%		\item \T{Development in respecting MVC pattern}{Développement de l'application en respectant le modèle MVC} ;
			%		\item \T{Management of pictures recording in PPM format thus scenes and points of view recording in XML format ;
			%				XML files syntax respecting a provided DTD file}
			%				{Gestion de l'enregistrement des images au format PPM ainsi que de la scène et des points de vues au format XML ;
			%				la syntaxe des fichiers XML respectant une DTD fournie} ;
			%		\item \T{Tests of the different functions with JUnit}{Élaboration de tests unitaires à l'aide de JUnit}.
			%	\end{myitemize}
			%}
	%}

\section{\textsc{\T{Languages}{Langues}}}

	\T{\cvlanguage{French}{Native speaker}{}}{}

	\cvlanguage{\T{English}{Anglais}}{\T{Working Proficient}{Niveau professionnel},
	\textbf{TOEIC 934}}{}

	%\cvlanguage{\T{Japanese \& Spanish}{Japonais \& Espagnol}}{\T{Elementary level}{Notion}}{}

\section{\textsc{\T{IT Skills}{Compétences}}}

	\cvline{\T{Programmings}{Programmation}}{C++, C, Java, Python, Ruby, Bash}
	\cvline{Web}{HTML, CSS, PHP, SQL, JavaScript}
	\cvline{Libraries / \T{Tools}{Outils}}{Android, QT, OpenCV, Buildroot, Boost, Gstreamer}
	\cvline{\T{Version control}{Gestion de version}}{Git, SVN}

\LongText{
\section{\textsc{\T{Interest}{Loisirs}}}

	\cvline{Sports}{\T{Regular training of swimming and course}{Natation, Course pédestre}.}
	\cvline{\T{Reading}{Lecture}}{\T{Fiction novel, international news}
	{Roman fiction, Actualité internationale}.}
}

\end{document}
